\documentclass[11pt]{article}
\usepackage{titlesec}
\usepackage{geometry}[margin=0.5in]
\usepackage{kpfonts, euler}
\usepackage[T1]{fontenc}
\usepackage{amsmath}

\titleformat{\section}[frame]
{\huge\bfseries}
{\thesection}
{1em}
{}

\titleformat{\subsection}
{\Large\bfseries}
{\thesubsection}
{1em}
{}[\titlerule]

\titleformat{\subsubsection}
{\bfseries\itshape}
{}
{0em}
{}

\title{Summary for Mathematical Methods in Physics}
\author{Namo}

\begin{document}
\section{Infinite Series}
\subsection{Convergence tests}
These are the important convergence tests. Another useful one is the p-series test. BEFORE using these tests, filter out series whose terms do not tend to zero because they are obviously divergent. We will denote our sum as $\sum_{}^{\infty} a_n$ .

\subsubsection{Comparison test}
First find a series whose convergence is known. Then compare the current series to that.

\subsubsection{Integral test}
This is straightforward. Just integrate with the upper limit as infinity:
$$\int_{}^{\infty} a_n \,dn \text{ is finite} \iff \sum_{}^{\infty} a_n \text{ converges} .$$

\subsubsection{Ratio test}
This is useful when the term contains factorial or power of $n$. Define $\rho$ as
$$\rho = \left| \frac{a_{n+1}}{a_n} \right| .$$
If $\lim_{n \to \infty} \rho$ is less than one, then the series converges; if more than 1, converges; if equals 1, the test is inconclusive.

\subsubsection{Limit comparison test}
This tells you if \textit{both} series converge or diverge. This is useful when the term is a rational number of polynomials. First find the comparison term $b_n$ by extracting the largest terms from the numerator and the denominator, then simplify the fraction.
\\
Then define
$$c = \lim_{n \to \infty} \frac{a_n}{b_n} .$$
If $0 < c < \infty$, then both converge or diverge. The convergence of the second series can be obtained through the p-series test.

\subsection{Power Series}
A power series is essentially a summation of different polynomial functions of $x$ that produces another function of $x$. It is of the form
\begin{align}
\sum_{n=0}^{\infty} a_n(x-a)^n &= a_0 + a_1(x-a) + a_2(x-a)^2 + a_3(x-a)^3 + ... \\
                               &= f_1(x) + f_2(x) + f_3(x) + f_4(x) + ... \\
                               &= S(x)
\end{align}

The coefficients depend on the incrementing $n$ while the $x$ terms have powers of $n$.
\subsubsection{Radius of convergence}
A power series converges differently from regular series. Since a power series is a function of $x$, there is a range of values of $x$ within which it converges. We also need to test both endpoints of the interval for convergence.
\subsubsection{Properties of power series}
\begin{itemize}
\item A power series can be differentiated/integrated term by term.
\item Two power series can be added, subtracted, or multiplied. This is simple when one of them is already a power series, e.g. $x^2$.
\item One series can be substituted into another if the values of the substituted series are in the interval of convergence of the other.
\item A function's power series expansion is unique.
\end{itemize}
\subsubsection{Applications}
\section{Linear Algebra}

\end{document}